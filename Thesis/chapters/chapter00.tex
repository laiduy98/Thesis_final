Neural activity in human's brain starts between the 17th and 23rd week of the prenatal development. It is believed that the electrical signal of the brain is not just reflecting the activity of the brain, but also the reflecting the activity of the whole body. This trust leads to the fact that people assume that if the brain signal can be recorded and analyze, scientists can further have more knowledge about the own human's body~\cite{EEGSignalProcessing}.

 Electroencephalography, or in a shorten word is EEG, is a method to record the neural oscillations (often known as "brain waves") of the brain and illustrate the operation of the body of human by that electrical signals. There are a lot of applications using this technique nowadays, mostly diagnose epilepsy, it is also used to diagnose sleep disorders and many more~\cite{EEG2}.
 
 Our objective is to characterize EEG signals according to the specific areas of the brain. We analyze the spatial-temporal patterns of the signals using machine learning tools. Given an EEG signal, the goal is to predict which part of the brain it belongs. To have such automatic tool can be very helpful for analyzing the plasticity of the brain, and useful for understanding how brain damages provoke changes in their functionality. It is known that when some area of the brain is damaged, the brain plasticity provokes that other parts can be activated and to have new functionalities in order of mitigating the impact of brain damages.

The goal of this thesis is to try and compare 2 popular machine learning techniques with different time steps to categorize 4 classes of EEG signal that provided. Each class will corresponding with a specific region of the brain.

The thesis is divided into 7 different chapters. In the second chapter, in order to have a basic idea of what and why we do this study, we will cover a theoretical background of the problem. Chapter 3 will cover the material that we use, chapter 4 and 5 is the preprocessing process and the machine method that we applied. Finally, in chapter 6 and 7, The results will be shown. We will have a conclusion and discuss about what to improve in the future.