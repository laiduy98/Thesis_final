EEG is one of the main diagnostic tests for epilepsy. A routine clinical EEG recording typically lasts 20–30 minutes (plus preparation time). It is a test that detects electrical activity in your brain using small, metal discs (electrodes) attached to your scalp. Routinely, EEG is used in clinical circumstances to determine changes in brain activity that might be useful in diagnosing brain disorders, especially epilepsy or another seizure disorder. An EEG might also be helpful for diagnosing or treating the following disorders:
\begin{itemize}
    \item Brain tumor
    \item Brain damage from head injury
    \item Brain dysfunction that can have a variety of causes (encephalopathy)
    \item Inflammation of the brain (encephalitis)
    \item Stroke
    \item Sleep disorders
\end{itemize}

It can also:
\begin{itemize}
    \item distinguish epileptic seizures from other types of spells, such as psychogenic non-epileptic seizures, syncope (fainting), sub-cortical movement disorders and migraine variants.
    \item differentiate "organic" encephalopathy or delirium from primary psychiatric syndromes such as catatonia.
    \item serve as an adjunct test of brain death in comatose patients.
    \item prognosticate in comatose patients (in certain instances).
    \item determine whether to wean anti-epileptic medications.
\end{itemize}

At times, a routine EEG is not sufficient to establish the diagnosis and/or to determine the best course of action in terms of treatment. In this case, attempts may be made to record an EEG while a seizure is occurring. This is known as an ictal recording, as opposed to an inter-ictal recording which refers to the EEG recording between seizures. To obtain an ictal recording, a prolonged EEG is typically performed accompanied by a time-synchronized video and audio recording. This can be done either as an outpatient (at home) or during a hospital admission, preferably to an Epilepsy Monitoring Unit (EMU) with nurses and other personnel trained in the care of patients with seizures. Outpatient ambulatory video EEGs typically last one to three days. An admission to an Epilepsy Monitoring Unit typically lasts several days but may last for a week or longer. While in the hospital, seizure medications are usually withdrawn to increase the odds that a seizure will occur during admission. For reasons of safety, medications are not withdrawn during an EEG outside of the hospital. Ambulatory video EEGs therefore have the advantage of convenience and are less expensive than a hospital admission, but the disadvantage of a decreased probability of recording a clinical event.

Epilepsy monitoring is typically done to distinguish epileptic seizures from other types of spells, such as psychogenic non-epileptic seizures, syncope (fainting), sub-cortical movement disorders and migraine variants, to characterize seizures for the purposes of treatment, and to localize the region of brain from which a seizure originates for work-up of possible seizure surgery.

Additionally, EEG may be used to monitor the depth of anesthesia, as an indirect indicator of cerebral perfusion in carotid endarterectomy, or to monitor amobarbital effect during the Wada test.

EEG can also be used in intensive care units for brain function monitoring to monitor for non-convulsive seizures/non-convulsive status epilepticus, to monitor the effect of sedative/anesthesia in patients in medically induced coma (for treatment of refractory seizures or increased intracranial pressure), and to monitor for secondary brain damage in conditions such as subarachnoid hemorrhage (currently a research method).

If a patient with epilepsy is being considered for resective surgery, it is often necessary to localize the focus (source) of the epileptic brain activity with a resolution greater than what is provided by scalp EEG. This is because the cerebrospinal fluid, skull and scalp smear the electrical potentials recorded by scalp EEG. In these cases, neurosurgeons typically implant strips and grids of electrodes (or penetrating depth electrodes) under the dura mater, through either a craniotomy or a burr hole. The recording of these signals is referred to as electrocorticography (ECoG), subdural EEG (sdEEG) or intracranial EEG (icEEG)--all terms for the same thing. The signal recorded from ECoG is on a different scale of activity than the brain activity recorded from scalp EEG. Low voltage, high frequency components that cannot be seen easily (or at all) in scalp EEG can be seen clearly in ECoG. Further, smaller electrodes (which cover a smaller parcel of brain surface) allow even lower voltage, faster components of brain activity to be seen. Some clinical sites record from penetrating microelectrodes.

Recent studies using machine learning techniques such as neural networks with statistical temporal features extracted from frontal lobe EEG brainwave data has shown high levels of success in classifying mental states (Relaxed, Neutral, Concentrating), mental emotional states (Negative, Neutral, Positive) and thalamocortical dysrhythmia.

EEG is not indicated for diagnosing headache. Recurring headache is a common pain problem, and this procedure is sometimes used in a search for a diagnosis, but it has no advantage over routine clinical evaluation.