All of the setup below is done inside Linux environment. You can find tutorial for Windows and other platform in website https://www.anaconda.com.

Step 1: Download Anaconda script
\begin{lstlisting}
cd /tmp
curl -O https://repo.anaconda.com/archive/Anaconda3-2019.03-Linux-x86_64.sh
\end{lstlisting}

\bigskip
Step 2: Run the Anaconda script
\begin{lstlisting}
bash Anaconda3-2019.03-Linux-x86_64.sh
\end{lstlisting}
You will receive the following output to review the license agreement by pressing ENTER until you reach the end.
\begin{lstlisting}
Welcome to Anaconda3 2019.03

In order to continue the installation process, please review the license
agreement.
Please, press ENTER to continue
>>>
...
Do you approve the license terms? [yes|no]
\end{lstlisting}
Type yes to agree with the license.
Now you have done with installing python.

\bigskip
Step 3: Install dependency
\begin{lstlisting}
conda install -c anaconda numpy
conda install -c anaconda pandas
conda install -c anaconda matplotlib
conda install -c conda-forge mne
conda install -c conda-forge keras
\end{lstlisting}
After done all the steps above, the system is good to go if you are ok using notepad to code python. But VSCode or Jupyter lab is a extremely good replacement with supported tools, for example, IntelliSense for auto complete the code.
\begin{lstlisting}
conda install -c conda-forge jupyterlab
\end{lstlisting}
With the VSCode, you have to add a repository.
\begin{lstlisting}
sudo apt updatesudo apt install software-properties-common apt-transport-https wget
wget -q https://packages.microsoft.com/keys/microsoft.asc -O- | sudo apt-key add -
sudo add-apt-repository "deb [arch=amd64] https://packages.microsoft.com/repos/vscode stable main"
sudo apt updatesudo apt install code
\end{lstlisting}
And you good to go.